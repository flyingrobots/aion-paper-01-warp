\documentclass{aion}

% ------------------------------------------------------------
% Metadata for this paper
% ------------------------------------------------------------
\renewcommand{\papertitle}{WARP Graphs: A Worldline Algebra for Recursive Provenance}
\renewcommand{\papernumber}{Paper I}
\renewcommand{\paperdate}{December 2025}

\renewcommand{\paperauthor}{James Ross}
\renewcommand{\paperaffiliation}{Independent Researcher}
\renewcommand{\paperorcid}{0009-0006-0025-7801}

% ------------------------------------------------------------
% Extra packages needed by this paper
% ------------------------------------------------------------
\usepackage{float}
% Macros for the WARPs paper
% Shared commands to keep notation consistent across the manuscript.
\usepackage{tikz} % Needed for \AIONLogo in this macros file
\usetikzlibrary{positioning,calc,shapes.geometric}
\newcommand{\AION}{\textrm{AI}\ensuremath{\Omega}\textrm{N}}
\newcommand{\AIONProjectURL}{\url{https://github.com/flyingrobots/aion}}

% Reusable TikZ diagrams for the AIΩN Foundations Series
% Requires: \usepackage{tikz} with libraries positioning,calc

% Diagram A — WARP skeleton with attachments
\newcommand{\DiagramWARPSkeleton}{
  \begin{tikzpicture}[>=stealth, node distance=2cm]
    \node[circle, draw, thick] (v1) {$v_1$};
    \node[circle, draw, thick, right of=v1] (v2) {$v_2$};
    \draw[->, thick] (v1) -- node[above] {$e$} (v2);
    \node[rectangle, draw, dashed, below=8pt of v1] (a1) {\small WARP};
    \draw[dashed] (v1) -- (a1);
    \node[rectangle, draw, dashed, below=8pt of v2] (a2) {\small WARP};
    \draw[dashed] (v2) -- (a2);
    \node[rectangle, draw, dashed, below=8pt of $(v1)!0.5!(v2)$] (ae) {\small WARP};
    \draw[dashed] ($(v1)!0.5!(v2)$) -- (ae);
  \end{tikzpicture}}

\newcommand{\AttachmentSep}{80pt}
\newcommand{\EdgeAttachmentSep}{68pt}

% Diagram A2 — Working example with payload labels
\newcommand{\DiagramCallGraphExample}{
  \begin{tikzpicture}[>=stealth, node distance=5.2cm, every node/.style={font=\small}]
    % Skeleton
    \node[circle, draw, very thick, inner sep=3pt] (vf) {$v_f$};
    \node[circle, draw, very thick, inner sep=3pt, right of=vf] (vg) {$v_g$};
    \draw[->, very thick] (vf) -- node[above=10pt] {$e_{\mathrm{call}}$} (vg);
    % Attachments
    \node[rectangle, draw, very thick, rounded corners=4pt, fill=blue!12, inner sep=5pt, below=\AttachmentSep of vf, xshift=-8pt] (astf) {AST($f$)};
    \node[rectangle, draw, very thick, rounded corners=4pt, fill=blue!12, inner sep=5pt, below=\AttachmentSep of vg, xshift=8pt] (astg) {AST($g$)};
    \node[rectangle, draw, very thick, rounded corners=6pt, fill=green!18, inner sep=6pt, below=\EdgeAttachmentSep of $(vf)!0.5!(vg)$] (prov) {Provenance / profiling};
    \draw[dashed, very thick] (vf.south) -- ++(0,-10pt) |- (astf.north);
    \draw[dashed, very thick] (vg.south) -- ++(0,-10pt) |- (astg.north);
    \draw[dashed, very thick] ($(vf)!0.5!(vg)$) -- ++(0,-10pt) |- (prov.north);
    % Nested WARPs inside attachments (to show recursion)
    \node[rectangle, draw, dashed, inner sep=3pt, below=16pt of astf] (astfsub) {\tiny $\bullet\!\!\!\to\!\!\!\bullet$};
    \draw[dashed, thick] (astf.south) -- (astfsub.north);
    \node[rectangle, draw, dashed, inner sep=3pt, below=16pt of astg] (astgsub) {\tiny $\bullet\!\!\!\leftarrow\!\!\!\bullet$};
    \draw[dashed, thick] (astg.south) -- (astgsub.north);
    \node[rectangle, draw, dashed, inner sep=3pt, below=16pt of prov] (provsub) {\tiny $\bullet\!-\!\bullet\!-\!\bullet$};
    \draw[dashed, thick] (prov.south) -- (provsub.north);
  \end{tikzpicture}}

\definecolor{LevelZeroAtom}{RGB}{90,90,90}
\definecolor{LevelZeroEdge}{RGB}{60,60,60}
\definecolor{LevelOneNode}{RGB}{160,200,255}
\definecolor{LevelOneEdge}{RGB}{40,110,200}
\definecolor{LevelTwoNode}{RGB}{255,182,193}
\definecolor{LevelTwoEdge}{RGB}{200,60,60}

% Diagram A1b — Tiny WARP: 3 vertices, 2 edges, atom attachments
\newcommand{\DiagramNestedWARP}{
  \begin{tikzpicture}[>=stealth, every node/.style={font=\small}]
    % Styles
    \tikzset{
      lvl2/.style={circle, draw=LevelTwoEdge, fill=LevelTwoNode!85, line width=1.8pt, minimum size=12mm},
      lvl2edge/.style={draw=LevelTwoEdge, line width=1.8pt},
      lvl1/.style={circle, draw=LevelOneEdge, fill=LevelOneNode!85, line width=1.5pt, minimum size=11mm},
      lvl1edge/.style={draw=LevelOneEdge, line width=1.5pt},
      atom/.style={circle, draw=LevelZeroEdge, fill=LevelZeroAtom!45, minimum size=5.5pt, inner sep=1pt},
      attachline/.style={dashed, thick, draw=black!70},
      note/.style={font=\scriptsize}
    }

    % Skeleton vertices (level 2)
    \node[lvl2,label=above:{\scriptsize $w_1$}] (w1) {};
    \node[lvl2,label=above:{\scriptsize $w_2$}, right=3.2cm of w1] (w2) {};
    \node[lvl2,label=above:{\scriptsize $w_3$}, right=3.2cm of w2] (w3) {};

    % Edges (level 2)
    \draw[lvl2edge,->] (w1) -- node[above, note] {$e_1$} (w2);
    \draw[lvl2edge,->] (w2) -- node[above, note] {$e_2$} (w3);

    % Atom attachments on w1, w3
    \node[atom, below=1.1cm of w1] (a1) {};
    \node[atom, below=1.1cm of w3] (a3) {};
    \draw[attachline] (w1) -- (a1);
    \draw[attachline] (w3) -- (a3);

    % Atom attachments on edges e1, e2
    \node[atom, below=1.0cm of $(w1)!0.5!(w2)$] (ae1) {};
    \node[atom, below=1.0cm of $(w2)!0.5!(w3)$] (ae2) {};
    \draw[attachline] ($(w1)!0.5!(w2)$) -- (ae1);
    \draw[attachline] ($(w2)!0.5!(w3)$) -- (ae2);

    % Nested WARP on w2 (level 1, blue)
    \node[lvl1, below=1.4cm of w2, label=above:{\scriptsize $u$}] (u) {};
    \node[lvl1, right=2.6cm of u, label=above:{\scriptsize $v$}] (v) {};
    \draw[lvl1edge,->] (u) -- node[above, note] {$e$} (v);

    % Atom attachments for the nested WARP
    \node[atom, below=1.0cm of u] (au) {};
    \node[atom, below=1.0cm of v] (av) {};
    \node[atom, below=1.0cm of $(u)!0.5!(v)$] (ae) {};
    \draw[attachline] (u) -- (au);
    \draw[attachline] (v) -- (av);
    \draw[attachline] ($(u)!0.5!(v)$) -- (ae);

    % Arrow showing w2 attaches the inner WARP
    \draw[attachline,->,>=stealth, ultra thick] (w2.south) -- ++(0,-0.3) |- (u.north);

    % Legend text
    \node[note, below=0.35cm of ae] {atoms};
    \node[note, above=0.25cm of u] {inner WARP $X_1$};
    \node[note, above=0.35cm of w2] {outer skeleton $X_2$};
  \end{tikzpicture}}

% Diagram A3 — Compiler pipeline snapshot
\newcommand{\DiagramCompilerPipeline}{
  \begin{tikzpicture}[>=stealth, node distance=2.6cm, every node/.style={font=\small}]
    % stages
    \node[rectangle, draw, very thick, rounded corners=3pt, fill=blue!8, inner sep=5pt] (p) {parse};
    \node[rectangle, draw, very thick, rounded corners=3pt, fill=blue!8, inner sep=5pt, right=2.6cm of p] (t) {typecheck};
    \node[rectangle, draw, very thick, rounded corners=3pt, fill=blue!8, inner sep=5pt, right=2.6cm of t] (o) {opt};
    \node[rectangle, draw, very thick, rounded corners=3pt, fill=blue!8, inner sep=5pt, right=2.6cm of o] (c) {codegen};
    \draw[->, very thick] (p) -- (t);
    \draw[->, very thick] (t) -- (o);
    \draw[->, very thick] (o) -- (c);
    % attachments under each stage
    \node[rectangle, draw, rounded corners=3pt, fill=green!18, inner sep=4pt, below=24pt of p] (patt) {AST};
    \node[rectangle, draw, rounded corners=3pt, fill=green!18, inner sep=4pt, below=24pt of t] (tatt) {typed core};
    \node[rectangle, draw, rounded corners=3pt, fill=green!18, inner sep=4pt, below=24pt of o] (oatt) {SSA CFG};
    \node[rectangle, draw, rounded corners=3pt, fill=green!18, inner sep=4pt, below=24pt of c] (catt) {target DAG};
    \draw[dashed, very thick] (p.south) -- (patt.north);
    \draw[dashed, very thick] (t.south) -- (tatt.north);
    \draw[dashed, very thick] (o.south) -- (oatt.north);
    \draw[dashed, very thick] (c.south) -- (catt.north);
    % edge provenance
    \node[rectangle, draw, rounded corners=3pt, fill=orange!20, inner sep=4pt, below=28pt of $(p)!0.5!(t)$] (prov1) {rewrite log};
    \node[rectangle, draw, rounded corners=3pt, fill=orange!20, inner sep=4pt, below=28pt of $(t)!0.5!(o)$] (prov2) {passes + timings};
    \node[rectangle, draw, rounded corners=3pt, fill=orange!20, inner sep=4pt, below=28pt of $(o)!0.5!(c)$] (prov3) {backend opts};
    \draw[dashed, very thick] ($(p)!0.5!(t)$) -- (prov1.north);
    \draw[dashed, very thick] ($(t)!0.5!(o)$) -- (prov2.north);
    \draw[dashed, very thick] ($(o)!0.5!(c)$) -- (prov3.north);
  \end{tikzpicture}}

% Diagram A4 — Microservice call with lineage
\newcommand{\DiagramMicroservice}{
  \begin{tikzpicture}[>=stealth, node distance=2.9cm, every node/.style={font=\small}]
    \node[rectangle, draw, very thick, rounded corners=3pt, fill=blue!8, inner sep=5pt] (fe) {frontend};
    \node[rectangle, draw, very thick, rounded corners=3pt, fill=blue!8, inner sep=5pt, right=2.9cm of fe] (au) {auth};
    \node[rectangle, draw, very thick, rounded corners=3pt, fill=blue!8, inner sep=5pt, right=2.9cm of au] (or) {orders};
    \node[rectangle, draw, very thick, rounded corners=3pt, fill=blue!8, inner sep=5pt, right=2.9cm of or] (bi) {billing};
    \draw[->, very thick] (fe) -- (au);
    \draw[->, very thick] (au) -- (or);
    \draw[->, very thick] (or) -- (bi);
    % attachments (local dataflow)
    \node[rectangle, draw, rounded corners=3pt, fill=green!18, inner sep=4pt, below=24pt of fe] (featt) {UI dataflow};
    \node[rectangle, draw, rounded corners=3pt, fill=green!18, inner sep=4pt, below=24pt of au] (auatt) {token check DAG};
    \node[rectangle, draw, rounded corners=3pt, fill=green!18, inner sep=4pt, below=24pt of or] (oratt) {order enrichment};
    \node[rectangle, draw, rounded corners=3pt, fill=green!18, inner sep=4pt, below=24pt of bi] (biatt) {charge calc};
    \draw[dashed, very thick] (fe.south) -- (featt.north);
    \draw[dashed, very thick] (au.south) -- (auatt.north);
    \draw[dashed, very thick] (or.south) -- (oratt.north);
    \draw[dashed, very thick] (bi.south) -- (biatt.north);
    % edge lineage
    \node[rectangle, draw, rounded corners=3pt, fill=orange!20, inner sep=4pt, below=28pt of $(fe)!0.5!(au)$] (l1) {trace ID + headers};
    \node[rectangle, draw, rounded corners=3pt, fill=orange!20, inner sep=4pt, below=28pt of $(au)!0.5!(or)$] (l2) {retries / spans};
    \node[rectangle, draw, rounded corners=3pt, fill=orange!20, inner sep=4pt, below=28pt of $(or)!0.5!(bi)$] (l3) {audit + latency};
    \draw[dashed, very thick] ($(fe)!0.5!(au)$) -- (l1.north);
    \draw[dashed, very thick] ($(au)!0.5!(or)$) -- (l2.north);
    \draw[dashed, very thick] ($(or)!0.5!(bi)$) -- (l3.north);
  \end{tikzpicture}}

% Diagram A5 — Robotics task plan
\newcommand{\DiagramRobotics}{
  \begin{tikzpicture}[>=stealth, node distance=3.2cm, every node/.style={font=\small}]
    \node[rectangle, draw, very thick, rounded corners=3pt, fill=blue!8, inner sep=5pt] (per) {perceive};
    \node[rectangle, draw, very thick, rounded corners=3pt, fill=blue!8, inner sep=5pt, right=3.2cm of per] (pla) {plan};
    \node[rectangle, draw, very thick, rounded corners=3pt, fill=blue!8, inner sep=5pt, right=3.2cm of pla] (act) {act};
    \draw[->, very thick] (per) -- (pla);
    \draw[->, very thick] (pla) -- (act);
    % attachments
    \node[rectangle, draw, rounded corners=3pt, fill=green!18, inner sep=4pt, below=28pt of per] (peratt) {sensor fusion DAG};
    \node[rectangle, draw, rounded corners=3pt, fill=green!18, inner sep=4pt, below=28pt of pla] (platt) {search tree};
    \node[rectangle, draw, rounded corners=3pt, fill=green!18, inner sep=4pt, below=28pt of act] (actatt) {motor seq + timing};
    \draw[dashed, very thick] (per.south) -- (peratt.north);
    \draw[dashed, very thick] (pla.south) -- (platt.north);
    \draw[dashed, very thick] (act.south) -- (actatt.north);
    % edge safety evidence
    \node[rectangle, draw, rounded corners=3pt, fill=orange!20, inner sep=4pt, below=32pt of $(per)!0.5!(pla)$] (s1) {perception certs};
    \node[rectangle, draw, rounded corners=3pt, fill=orange!20, inner sep=4pt, below=32pt of $(pla)!0.5!(act)$] (s2) {collision checks};
    \draw[dashed, very thick] ($(per)!0.5!(pla)$) -- (s1.north);
    \draw[dashed, very thick] ($(pla)!0.5!(act)$) -- (s2.north);
  \end{tikzpicture}}

% Diagram A6 — Formal refinement / proofs (3 levels deep)
\newcommand{\DiagramRefinement}{
  \begin{tikzpicture}[>=stealth, node distance=3.2cm, every node/.style={font=\small}]
    % skeleton
    \node[rectangle, draw, very thick, rounded corners=3pt, fill=blue!8, inner sep=5pt] (spec) {spec};
    \node[rectangle, draw, very thick, rounded corners=3pt, fill=blue!8, inner sep=5pt, right=3.2cm of spec] (model) {model};
    \node[rectangle, draw, very thick, rounded corners=3pt, fill=blue!8, inner sep=5pt, right=3.2cm of model] (code) {code};
    \draw[->, very thick] (spec) -- (model);
    \draw[->, very thick] (model) -- (code);
    % level-1 attachments
    \node[rectangle, draw, rounded corners=3pt, fill=green!18, inner sep=4pt, below=28pt of spec] (specatt) {LTS / contracts};
    \node[rectangle, draw, rounded corners=3pt, fill=green!18, inner sep=4pt, below=28pt of model] (modatt) {state machine};
    \node[rectangle, draw, rounded corners=3pt, fill=green!18, inner sep=4pt, below=28pt of code] (codeatt) {CFG / SSA};
    \draw[dashed, very thick] (spec.south) -- (specatt.north);
    \draw[dashed, very thick] (model.south) -- (modatt.north);
    \draw[dashed, very thick] (code.south) -- (codeatt.north);
    % edge attachments (proof artifacts)
    \node[rectangle, draw, rounded corners=3pt, fill=orange!20, inner sep=4pt, below=32pt of $(spec)!0.5!(model)$] (proof1) {refinement proof};
    \node[rectangle, draw, rounded corners=3pt, fill=orange!20, inner sep=4pt, below=32pt of $(model)!0.5!(code)$] (proof2) {verification conditions};
    \draw[dashed, very thick] ($(spec)!0.5!(model)$) -- (proof1.north);
    \draw[dashed, very thick] ($(model)!0.5!(code)$) -- (proof2.north);
    % level-2 attachments under model attachment (deep nesting)
    \node[rectangle, draw, rounded corners=3pt, fill=yellow!25, inner sep=3pt, below=24pt of modatt] (ctrex) {counterexample trace};
    \draw[dashed, thick] (modatt.south) -- (ctrex.north);
    % level-2 under proof artifact
    \node[rectangle, draw, rounded corners=3pt, fill=yellow!25, inner sep=3pt, below=24pt of proof2] (lemmas) {lemma DAG};
    \draw[dashed, thick] (proof2.south) -- (lemmas.north);
  \end{tikzpicture}}

% Diagram A7 — Self-hosting compiler bootstrap (4 levels deep)
\newcommand{\DiagramBootstrapCompiler}{
  \begin{tikzpicture}[>=stealth, node distance=2.7cm, every node/.style={font=\small}]
    % skeleton (build stages)
    \node[rectangle, draw, very thick, rounded corners=3pt, fill=blue!8, inner sep=5pt] (p) {parse};
    \node[rectangle, draw, very thick, rounded corners=3pt, fill=blue!8, inner sep=5pt, right=2.7cm of p] (t) {typecheck};
    \node[rectangle, draw, very thick, rounded corners=3pt, fill=blue!8, inner sep=5pt, right=2.7cm of t] (o) {opt};
    \node[rectangle, draw, very thick, rounded corners=3pt, fill=blue!8, inner sep=5pt, right=2.7cm of o] (c) {codegen};
    \node[rectangle, draw, very thick, rounded corners=3pt, fill=blue!8, inner sep=5pt, right=2.7cm of c] (l) {link};
    \draw[->, very thick] (p) -- (t);
    \draw[->, very thick] (t) -- (o);
    \draw[->, very thick] (o) -- (c);
    \draw[->, very thick] (c) -- (l);
    % Level 1: IR attachments (twice the spacing)
    \node[rectangle, draw, rounded corners=3pt, fill=green!18, inner sep=4pt, below=28pt of p] (p1) {AST};
    \node[rectangle, draw, rounded corners=3pt, fill=green!18, inner sep=4pt, below=28pt of t] (t1) {typed core};
    \node[rectangle, draw, rounded corners=3pt, fill=green!18, inner sep=4pt, below=28pt of o] (o1) {SSA CFG};
    \node[rectangle, draw, rounded corners=3pt, fill=green!18, inner sep=4pt, below=28pt of c] (c1) {backend DAG};
    \node[rectangle, draw, rounded corners=3pt, fill=green!18, inner sep=4pt, below=28pt of l] (l1) {ELF graph};
    \draw[dashed, very thick] (p.south) -- (p1.north);
    \draw[dashed, very thick] (t.south) -- (t1.north);
    \draw[dashed, very thick] (o.south) -- (o1.north);
    \draw[dashed, very thick] (c.south) -- (c1.north);
    \draw[dashed, very thick] (l.south) -- (l1.north);
    % Level 2: provenance per stage
    \node[rectangle, draw, rounded corners=3pt, fill=orange!20, inner sep=4pt, below=32pt of p1] (p2) {parse log};
    \node[rectangle, draw, rounded corners=3pt, fill=orange!20, inner sep=4pt, below=32pt of t1] (t2) {typing proofs};
    \node[rectangle, draw, rounded corners=3pt, fill=orange!20, inner sep=4pt, below=32pt of o1] (o2) {opt passes};
    \node[rectangle, draw, rounded corners=3pt, fill=orange!20, inner sep=4pt, below=32pt of c1] (c2) {regalloc trace};
    \node[rectangle, draw, rounded corners=3pt, fill=orange!20, inner sep=4pt, below=32pt of l1] (l2) {link map};
    \draw[dashed, very thick] (p1.south) -- (p2.north);
    \draw[dashed, very thick] (t1.south) -- (t2.north);
    \draw[dashed, very thick] (o1.south) -- (o2.north);
    \draw[dashed, very thick] (c1.south) -- (c2.north);
    \draw[dashed, very thick] (l1.south) -- (l2.north);
    % Level 3: proof/CE artifacts under provenance
    \node[rectangle, draw, rounded corners=3pt, fill=yellow!25, inner sep=3pt, below=32pt of t2] (t3) {SMT certs};
    \node[rectangle, draw, rounded corners=3pt, fill=yellow!25, inner sep=3pt, below=32pt of o2] (o3) {inlining CE traces};
    \node[rectangle, draw, rounded corners=3pt, fill=yellow!25, inner sep=3pt, below=32pt of c2] (c3) {allocator proofs};
    \node[rectangle, draw, rounded corners=3pt, fill=yellow!25, inner sep=3pt, below=32pt of l2] (l3) {checksum chain};
    \draw[dashed, thick] (t2.south) -- (t3.north);
    \draw[dashed, thick] (o2.south) -- (o3.north);
    \draw[dashed, thick] (c2.south) -- (c3.north);
    \draw[dashed, thick] (l2.south) -- (l3.north);
    % Level 4: nested WARPs under selected artifacts (showing recursion)
    \node[rectangle, draw, dashed, inner sep=3pt, below=10pt of o3] (o4) {\tiny $\bullet\!\to\!\bullet\!\leftarrow\!\bullet$};
    \draw[dashed, thick] (o3.south) -- (o4.north);
    \node[rectangle, draw, dashed, inner sep=3pt, below=10pt of c3] (c4) {\tiny $\bullet\!-\!\bullet\!-\!\bullet$};
    \draw[dashed, thick] (c3.south) -- (c4.north);
  \end{tikzpicture}}

% Diagram B — Finite unfolding tower
\newcommand{\DiagramUnfoldingTower}{
  \begin{tikzpicture}[node distance=1.8cm]
    \node (x0) [rectangle, draw] {$\mathrm{unf}_0(X)$};
    \node (x1) [rectangle, draw, below of=x0] {$\mathrm{unf}_1(X)$};
    \node (x2) [rectangle, draw, below of=x1] {$\mathrm{unf}_2(X)$};
    \node (xinfty) [rectangle, draw, below of=x2] {$\mathrm{unf}_\infty(X)$};
    \draw[->] (x0) -- (x1);
    \draw[->] (x1) -- (x2);
    \draw[->] (x2) -- (xinfty);
  \end{tikzpicture}}

% Diagram C — Morphism of WARPs
\newcommand{\DiagramWARPmorphism}{
  \begin{tikzpicture}[>=stealth, node distance=2cm]
    \node[circle, draw] (Xv1) {$v_1$};
    \node[circle, draw, right of=Xv1] (Xv2) {$v_2$};
    \draw[->] (Xv1) -- (Xv2);
    \node[circle, draw, below=2cm of Xv1] (Yv1) {$v'_1$};
    \node[circle, draw, right of=Yv1] (Yv2) {$v'_2$};
    \draw[->] (Yv1) -- (Yv2);
    \draw[->, thick] (Xv1) -- (Yv1);
    \draw[->, thick] (Xv2) -- (Yv2);
  \end{tikzpicture}}

% Diagram D — Embedding ordinary graph
\newcommand{\DiagramGraphEmbedding}{
  \begin{tikzpicture}[>=stealth, node distance=2.2cm]
    \node[circle,draw,minimum size=0.7cm] (a) {$a$};
    \node[circle,draw,minimum size=0.7cm,right of=a] (b) {$b$};
    \draw[->] (a) -- node[above] {$e$} (b);
    \node[below=0.8cm of a] (aa) {\footnotesize $\Atom(p_\bullet)$};
    \node[below=0.8cm of b] (bb) {\footnotesize $\Atom(p_\bullet)$};
    \node at ($(a)!0.5!(b)+(0,-1.2)$) {\footnotesize $\Atom(p_\bullet)$ on $e$};
    \draw[dashed] (a) -- (aa);
    \draw[dashed] (b) -- (bb);
    \draw[dashed] ($(a)!0.5!(b)$) -- ($(a)!0.5!(b)+(0,-0.6)$);
  \end{tikzpicture}}


% ------------------------------------------------------------
% Notation shortcuts (guarded to avoid clashes)
% ------------------------------------------------------------
\DeclareRobustCommand{\WARP}{\ifmmode\mathit{WARP}\else\textit{WARP}\fi}
\ifdefined\Atom\else\newcommand{\Atom}{\operatorname{Atom}}\fi
\ifdefined\depthOf\else\DeclareMathOperator{\depthOf}{depth}\fi
\ifdefined\OGraph\else\newcommand{\OGraph}{\mathbf{OGraph}}\fi
\ifdefined\OGraphT\else\newcommand{\OGraphT}{\mathbf{OGraph}_T}\fi
\ifdefined\Graph\else\newcommand{\Graph}{\mathbf{Graph}}\fi
% Give hyperref unique anchors for figures to silence duplicate-destination warnings
\makeatletter
\renewcommand{\theHfigure}{\thesection.\arabic{figure}}
\makeatother

\begin{document}

\AIONTitlePage

\cleardoublepage
\thispagestyle{empty}

\begin{center}
{\small
\vspace*{\fill}

\textcopyright~2025 James Ross\\[0.75em]

This work is licensed under the\\
\textbf{Creative Commons Attribution 4.0 International License (CC BY 4.0).}\\[1em]

You are free to:
\begin{itemize}[leftmargin=2em]
  \item \textbf{Share} — copy and redistribute the material in any medium or format
  \item \textbf{Adapt} — remix, transform, and build upon the material for any purpose
\end{itemize}

Under the following terms:
\begin{itemize}[leftmargin=2em]
  \item \textbf{Attribution} — You must give appropriate credit, provide a link to the license,\\
        and indicate if changes were made. You may not imply endorsement by the author.
\end{itemize}

For the full legal code of the license, see:\\[0.25em]
\url{https://creativecommons.org/licenses/by/4.0/}

\vspace*{\fill}
}
\end{center}

\cleardoublepage

\begin{abstract}
Many computational systems are best described as graphs of graphs of graphs.
This paper introduces WARP graphs (Worldline Algebra for Recursive Provenance), a minimal categorical
structure that captures nested graph-shaped state. Formally, a WARP graph is a finite directed multigraph whose vertices and edges carry attached WARPs, yielding a finite well-founded hierarchy. The ``worldline'' terminology anticipates the deterministic evolution of such states developed in subsequent papers; this paper focuses on the static state object. WARPs unify hierarchical system structure, syntax, control
flow, provenance, and traces into one algebraic
object.  We give an inductive definition, an initial-algebra characterisation,
a category of morphisms, and embeddings of ordinary graphs and hypergraphs.
This paper forms the structural foundation for the \AION{} series, enabling
deterministic rewriting, holographic provenance, and observer geometry in
later developments.
\end{abstract}

\clearpage
\tableofcontents
\clearpage

\section{Introduction}
\label{sec:intro}

Complex software does not live in a single flat graph.  A compiler juggles
syntax trees, control-flow graphs, and optimisation traces.  A database query
planner tracks schemas, query plans, and execution traces.  An AI agent may
carry a world model, internal goals, and a ledger of prior interactions.  In
practice, engineers assemble these layers manually, creating ad-hoc ``graphs
of graphs'' conventions that resist generalisation.

The usual mathematical tools---directed graphs, hypergraphs, typed open
graphs---are excellent for flat structure, but awkward for nested structure.
Every project invents its own ad hoc notion of ``graph with attached
subgraphs'', usually encoded as a pile of pointers and conventions.  That makes
it hard to transport results from one setting to another, and even harder to
state semantics that talk about the whole stack at once.

This paper extracts a minimal canonical object for this job: \emph{WARP graphs} (Worldline Algebra for Recursive Provenance).  Earlier drafts and companion notes used the name \emph{recursive metagraphs} (RMGs); we adopt \emph{WARP} to avoid collisions with existing uses of ``RMG'' and to emphasise the provenance/worldline interpretation.  Informally, you take an ordinary finite
directed multigraph (the \emph{skeleton}) and allow each vertex and edge to
carry its own attached WARP graph, and those attachments may themselves have
attachments, and so on, but with the guarantee that this nesting is finite and
well-founded.

We call an individual object a \emph{WARP graph}; when unambiguous, simply a \emph{WARP} (plural \emph{WARPs}).

There are many formalisms in the same neighbourhood (nested graphs, bigraphs,
hierarchical state machines, etc.).  The design choice here is ruthlessly
pragmatic:

\begin{itemize}
  \item use as little structure as possible beyond standard typed graphs;
  \item make the construction compatible with adhesive categories and DPO
        graph rewriting~\cite{EL97,EEPT06,LS08}, so that rewriting extends cleanly;
  \item support both an inductive definition and an initial-algebra view, so
        that structural recursion and induction are available out of the box.
\end{itemize}

Everything else in the \AION{} Foundations series sits on top of this
object.  Later papers will use WARPs as the state space for deterministic
multiway rewriting, holographic provenance, rulial distance between observers,
and operating-system-level architecture.  Those topics are deliberately kept
offstage here; this paper focuses entirely on the definition and basic
categorical properties of WARPs.

\paragraph{Acknowledgments and prior art.}
The graph-theoretic and categorical foundations underlying WARP graphs draw on a long line of work in algebraic graph transformation, including the double-pushout approach and adhesive categories developed by Ehrig, Prange, Taentzer, Lack, Soboci\'nski, and others. Related formalisms for nested or hierarchical graph structure include bigraphs, hierarchical state machines, and various ``graph-of-graphs'' encodings used in program analysis and heterogeneous network theory.

The contribution of this work is not the invention of graph rewriting or nesting per se, but the isolation of a single initial-algebraic state object intended to serve as a common substrate for deterministic multiway semantics, explicit provenance payloads, and observer geometry developed in subsequent papers.
Foundational references include standard treatments of algebraic graph
transformation~\cite{EL97,EEPT06} and adhesive categories~\cite{LS08}, as well as
Milner's bigraphs~\cite{Mil02}.

\paragraph{Contributions.}
The specific contributions are:

\begin{enumerate}
  \item A precise inductive definition of WARP graphs (Worldline Algebra for Recursive Provenance) as the least
        class generated by atomic payloads and decorated skeleton graphs.

  \item An equivalent initial-algebra presentation via a finitary polynomial
        endofunctor, yielding canonical recursion and induction schemes.

  \item A notion of morphism making WARPs into a category $\mathbf{WARP}$, together
        with a forgetful functor to the category of finite directed
        multigraphs.

  \item A relation to standard models: ordinary typed open graphs and
        hypergraphs embed fully and faithfully into suitable WARP graphs, so existing
        graph-based semantics extend without change.
\end{enumerate}

\paragraph{Structure of the paper.}
Section~\ref{sec:WARP} introduces the two-plane view of WARP graphs.
Section~\ref{sec:WARP-inductive} gives the inductive definition and a worked
example.  Section~\ref{sec:WARP-initial} recasts WARPs as an initial algebra.
Section~\ref{sec:WARP-unfold} introduces depth and finite unfoldings.
Section~\ref{sec:WARP-morphisms} defines morphisms and the category
$\mathbf{WARP}$.  Section~\ref{sec:WARP-relation} relates WARP graphs to ordinary
graphs and hypergraphs.  Section~\ref{sec:WARP-outlook} briefly situates WARPs in
the broader \AION{} programme.  Section~\ref{sec:WARP-notation} summarises the
main notation.

\clearpage

\section{Worldline Algebra for Recursive Provenance Graphs}
\label{sec:WARP}

In this section we define Worldline Algebra for Recursive Provenance graphs and relate them to standard
graph models.  Throughout, a \emph{directed multigraph} $S$ is a tuple
$(V_S, E_S, s_S, t_S)$ with finite sets of vertices $V_S$ and edges $E_S$ and
source and target maps
\[
  s_S, t_S : E_S \to V_S.
\]
We write $V(S)$ and $E(S)$ when the underlying graph is clear from context.

\paragraph{Remark.}
Although we state skeletons as plain finite multigraphs for concreteness, the recursive construction lifts verbatim to richer skeleton categories (for example typed open graphs $\OGraphT$).  We will exploit this generality when developing rewriting and provenance semantics on WARPs in later papers.

Intuitively, a WARP graph has two layers:

\begin{itemize}
  \item a finite skeleton graph $S$ that describes the coarse shape of the
        state; and
  \item an \emph{attachment} at each vertex and edge, which is itself a WARP graph,
        capturing nested structure or payload.
\end{itemize}

Iterating attachments produces a finite, well-founded tree of graphs.  The rest
of the section spells this out.

\subsection{Inductive definition}
\label{sec:WARP-inductive}

Fix a set $P$ of \emph{atomic payloads}: literals, external identifiers, or
other opaque data items.  Think of $P$ as the stuff we are not going to model
internally (bytestrings, floats, external object IDs, \dots).

We assume $P$ contains a distinguished placeholder payload $p_\bullet$, used when we need a generic atom (e.g., in truncations and shallow embeddings).

\begin{definition}[Worldline Algebra for Recursive Provenance Graph]\label{def:warp-graph}
The class $\WARP$ of Worldline Algebra for Recursive Provenance graphs is the least class closed under the
following two constructors:
\begin{enumerate}
  \item For each $p \in P$ there is an \emph{atom} $\Atom(p) \in \WARP$.

  \item If $S = (V_S, E_S, s_S, t_S)$ is a finite directed multigraph and
        $\alpha : V_S \to \WARP$ and $\beta : E_S \to \WARP$ are attachment
        assignments, then the triple
        \[
          (S, \alpha, \beta) \in \WARP.
        \]
\end{enumerate}
We call $S$ the \emph{skeleton} and $\alpha$ and $\beta$ the \emph{vertex} and
\emph{edge} attachments.
\end{definition}

This is a standard well-founded inductive definition: every WARP is either a
bare atom or a finite skeleton whose vertices and edges carry smaller WARPs.
Because the construction is least, every WARP can be \emph{built up} from atoms
by finitely many applications of the second clause.

\begin{example}[Program call graph with attached syntax and provenance]\label{ex:one}
Consider a tiny program with functions $f$ and $g$ and a single call
$f \to g$.  The skeleton graph $S$ has two vertices $v_f, v_g$ and one edge
$e_{\mathrm{call}} : v_f \to v_g$.  Attach to $v_f$ and $v_g$ the abstract
syntax trees of the corresponding functions, and attach to $e_{\mathrm{call}}$
a small provenance graph recording optimisation choices or profiling data.

This yields a WARP $(S, \alpha, \beta)$ with
\[
  V_S = \{v_f, v_g\},\qquad E_S = \{e_{\mathrm{call}}\},\qquad
  s_S(e_{\mathrm{call}})=v_f,\; t_S(e_{\mathrm{call}})=v_g.
\]
Attachments: $\alpha(v_f)$ and $\alpha(v_g)$ are the respective abstract syntax
trees; $\beta(e_{\mathrm{call}})$ is a tiny provenance graph (e.g., a sequence
of optimisation passes).  All three attachments can themselves carry attachments
if, say, a syntax tree node needs to store profiling WARPs.  In one object the
high-level call graph and nested payloads stay coherent.

The figure shows this recursion: dashed mini-graphs hang off each attachment to
remind the reader that attachments are full WARPs, not mere labels.
\end{example}

\begin{figure}[H]
  \centering
  \DiagramCallGraphExample
  \caption{Worked example: call-graph skeleton with attached ASTs for $f$ and $g$, and provenance payload on the call edge. Dashed mini-graphs inside attachments make the recursive nesting explicit.}
  \label{fig:WARP-structure}
\end{figure}

It is often convenient to write a WARP informally as
\[
  U = (G; \alpha, \beta),
\]
with $G$ emphasising the skeleton and $(\alpha, \beta)$ the attachments.

\subsection{Initial algebra viewpoint}
\label{sec:WARP-initial}

The inductive definition above already gives us a notion of recursion, but it is
useful to rephrase WARPs as an initial algebra for a polynomial functor.  This
makes structural recursion and induction theorems essentially automatic.

Let $\mathcal{G}$ be a small set of isomorphism-class representatives of finite
directed multigraphs (the allowed skeleton shapes).  Define a finitary
endofunctor
\[
  F : \mathbf{Set} \to \mathbf{Set}
\]
by
\[
  F(X) \;=\; P \;\;+\!\!\!\!\sum_{S \in \mathcal{G}}\!\! (V_S \to X) \times (E_S \to X).
\]
An element of $F(X)$ is either an atom in $P$ or a choice of skeleton shape
$S \in \mathcal{G}$ together with an $X$-label for each vertex and edge of $S$.

\begin{proposition}\label{prop:one}
Up to isomorphism, $\WARP$ is the carrier of the initial $F$-algebra.
\end{proposition}

Sketching the idea: the inductive clauses for $\WARP$ are exactly the clauses of
an $F$-algebra structure $\mathit{in} : F(\WARP) \to \WARP$.  Conversely, given any
$F$-algebra $(X, a)$, there is a unique \emph{fold}
$\mathit{fold}_a : \WARP \to X$ that interprets atoms and skeletons according to
$a$.  This is the usual initial-algebra story for polynomial functors.

Practically, this tells you that to define a function out of WARPs, it suffices
to say:

\begin{itemize}
  \item how it acts on atoms, and
  \item given a skeleton $S$ and recursively computed results for all
        attachments, how to combine those into a result for $(S, \alpha, \beta)$.
\end{itemize}

The result is then unique.  We will quietly rely on this pattern when defining
operations on WARPs later.

\section{Depth and Unfoldings}
\label{sec:WARP-unfold}

Although WARPs are finite by definition, the attachment tree can be deep.  It is
often useful to reason about them via finite-depth approximations.

\begin{definition}[Depth]\label{def:depth}
The \emph{depth} of a WARP $X$ is the length of the longest attachment chain in
$X$.  Atoms have depth~0.  If $X = (S, \alpha, \beta)$, its depth is
\[
  \max\bigl\{ 1 + \depthOf(\alpha(v)), \; 1 + \depthOf(\beta(e))
    \mid v \in V_S, e \in E_S \bigr\}.
\]
\end{definition}

For $k \in \mathbb{N}$, define the \emph{$k$-unfolding} $\mathsf{unf}_k(X)$ to
be the WARP obtained by:

\begin{itemize}
  \item keeping all structure at attachment depths $0,\dots,k-1$ unchanged; and
  \item replacing every attachment at depth $\ge k$ by the distinguished placeholder atom $\Atom(p_\bullet)$.
\end{itemize}

This can be defined by structural recursion on $X$.  Each $\mathsf{unf}_k(X)$
is a finite WARP, and for $k \le \ell$ there is a natural embedding
$\mathsf{unf}_k(X) \to \mathsf{unf}_\ell(X)$.

The \emph{infinite unfolding} $\mathsf{unf}_\infty(X)$ is the directed colimit
of the diagram
\[
  \mathsf{unf}_0(X) \to \mathsf{unf}_1(X) \to \mathsf{unf}_2(X) \to \cdots
\]
in $\mathbf{Set}$.  In general this colimit is an infinite attachment tree, so
it need not itself be a WARP in our finite class $\WARP$.  Intuitively it
records, for each finite depth, what $X$ looks like up to that depth.  Every
finite pattern that occurs in $X$ appears in some finite unfolding, so reasoning
about $X$ can often be reduced to reasoning about all its approximants.

We will not need any stronger universal property of $\mathsf{unf}_\infty(X)$ in
this paper, but the picture is useful when thinking about observers and
multiway systems in later work.

\begin{figure}[ht]
  \centering
  \DiagramUnfoldingTower
  \caption{Depth and unfoldings: successive $\mathsf{unf}_k(X)$ approximations converge to $\mathsf{unf}_\infty(X)$.}
  \label{fig:depth-unfold}
\end{figure}

\section{Morphisms and the Category of WARPs}
\label{sec:WARP-morphisms}

To use WARPs as honest mathematical objects, we need a notion of morphism.

First regard the set $P$ of atomic payloads as a discrete category $\mathbf{P}$:
objects are elements of $P$, the only morphisms are identities.  We embed
atoms into our category faithfully: there are no non-trivial maps between
distinct atoms.

\begin{definition}[WARP morphism]\label{def:WARP-morphism}
A \emph{morphism} $f : X \to Y$ of WARPs is defined by structural
recursion on depth.

\begin{itemize}
  \item On atoms: a morphism $\Atom(p) \to \Atom(p')$ exists iff $p = p'$, in
        which case it is the identity.

  \item On composite objects: for
        $X = (S, \alpha, \beta)$ and $Y = (S', \alpha', \beta')$, a morphism
        $f : X \to Y$ consists of
        \begin{enumerate}
          \item a graph homomorphism of skeletons
                \[
                  f_V : V_S \to V_{S'}, \qquad f_E : E_S \to E_{S'}
                \]
                preserving sources and targets; and

          \item for every vertex $v \in V_S$ a morphism of attachments
                $f_v : \alpha(v) \to \alpha'(f_V(v))$, and for every edge
                $e \in E_S$ a morphism
                $f_e : \beta(e) \to \beta'(f_E(e))$.
        \end{enumerate}
\end{itemize}

Composition and identities are defined componentwise.
\end{definition}

The recursion in the second clause is well-founded because attachments have
strictly smaller depth than the composite they sit in.

\begin{figure}[ht]
  \centering
  \DiagramWARPmorphism
  \caption{A WARP morphism consists of a skeleton map $(f_V,f_E)$ plus attachment morphisms on every vertex and edge.}
  \label{fig:morphism}
\end{figure}

\begin{proposition}\label{prop:two}
WARPs and their morphisms form a category, written
$\mathbf{WARP}$.
\end{proposition}

There is a forgetful functor
\[
  \pi : \mathbf{WARP} \to \Graph
\]
sending $(S, \alpha, \beta)$ to its skeleton $S$, sending each atom $\Atom(p)$ to the empty graph, and acting on morphisms by forgetting the attachment components.  This functor will later be refined to a
fibration when we talk about rewriting on the ``skeleton plane'' and
``attachment planes''; for now it simply records that every WARP lives over an
ordinary finite directed multigraph.

\section{Relation to Ordinary Graphs and Hypergraphs}
\label{sec:WARP-relation}

WARPs are intended to \emph{extend} rather than replace standard graph models.
Two simple embeddings are particularly useful.

\paragraph{Ordinary graphs as shallow WARPs.}
Fix a set of atomic payloads $P$ with at least one distinguished element
$p_\bullet$.  Every finite directed multigraph $S$ can be viewed as a shallow WARP graph by attaching the constant atom $\Atom(p_\bullet)$ to every vertex and
edge:
\[
  \iota_{\mathrm{graph}}(S)
    = (S, \alpha, \beta)
\]
with $\alpha(v) = \Atom(p_\bullet)$ and
$\beta(e) = \Atom(p_\bullet)$ for all $v,e$.  Morphisms of graphs lift in the
obvious way.  This gives a fully faithful embedding of the category of finite
directed multigraphs into $\mathbf{WARP}$ as the subcategory of depth~$1$
objects with trivial attachments.

\begin{figure}[ht]
  \centering
  \DiagramGraphEmbedding
  \caption{Embedding an ordinary multigraph by attaching a constant atom to every vertex and edge.}
  \label{fig:graph-embedding}
\end{figure}

\paragraph{Hypergraphs via typed open graphs.}
Let $\OGraphT$ be a category of $T$-typed open graphs as in
Lack--Soboci\'nski~\cite{LS08}: objects are cospans $I \hookrightarrow G \hookleftarrow O$
of monomorphisms and morphisms are commuting maps of cospans.  This category is
adhesive, and typed hypergraphs embed fully and faithfully via the usual
incidence construction~\cite{EL97,EEPT06}.  In particular, DPO rewriting is
well-behaved in $\OGraphT$.

We can regard a WARP graph whose skeleton lies in $\OGraphT$ as a ``recursive typed
open graph'': the base graph has open inputs and outputs, and attachments live
in the fibres over nodes and edges.  The forgetful functor
$\pi : \mathbf{WARP} \to \OGraphT$ simply drops attachments.  In the companion
paper on deterministic multiway semantics, we will exploit this to lift DPO
rewriting from skeletons to full WARP states.

For the present paper, the key point is that WARPs \emph{subsume} ordinary typed
open graphs and hypergraphs while adding recursively nested structure.  Any
model expressible in the usual DPO setting can be expressed as a shallow WARP;
models that genuinely need nested graphs get additional expressive power with
no change to the underlying graph-theoretic machinery.  Related nested-graph
formalisms such as bigraphs~\cite{Mil02} offer alternative semantics; WARPs
are intentionally minimal while remaining adhesive-friendly.

\section{Discussion and Outlook}
\label{sec:WARP-outlook}

WARPs provide a minimal, well-founded ``graphs all the way down'' substrate that
is compatible with adhesive graph rewriting.  This paper isolates that object
and its basic categorical properties so that later work can treat WARPs as the
canonical state space for nested computational systems.

In the companion papers in the \AION{} Foundations series, we will build on this
substrate to define deterministic multiway DPO rewriting on WARPs, show how
boundary provenance can serve as a holographic encoding of interior evolution,
and introduce an MDL-based geometry on observers (rulial distance) over WARP
universes.  The present definitions are designed precisely so those extensions
lift from skeletons to full recursive states without changing the underlying
graph-theoretic machinery.

\section{Notation Summary}
\label{sec:WARP-notation}

For reference, we collect the main notation used in this paper.

\begin{center}
\begin{tabular}{ll}
  Symbol & Meaning \\ \hline
  $P$ & set of atomic payloads \\
  $p_\bullet$ & distinguished placeholder payload in $P$ \\
  $\Atom(p)$ & atomic WARP corresponding to $p \in P$ \\
  $S = (V_S, E_S, s_S, t_S)$ & finite directed multigraph (skeleton) \\
  $(S, \alpha, \beta)$ & WARP with vertex and edge attachments \\
  $\WARP$ & class of all WARP graphs \\
  $\mathbf{WARP}$ & category of WARP graphs and morphisms \\
  $\depthOf(X)$ & depth of a WARP graph $X$ \\
  $\mathsf{unf}_k(X)$ & $k$-unfolding (finite-depth approximation) \\
  $\mathsf{unf}_\infty(X)$ & colimit of the unfolding tower of $X$ \\
  $\pi : \mathbf{WARP} \to \Graph$ & forgetful functor to skeleton graphs
\end{tabular}
\end{center}

\clearpage
\appendix

\section{Appendix: Working Example}

\begin{figure}[H]
  \centering
  \DiagramNestedWARP
  \caption{Nested WARP Graphs: the middle vertex $w_2$ carries the inner WARP graph $X_1$; all other attachments are atoms.}
  \label{fig:appendix-nested}
\end{figure}

\subsection{Tiny explicit nested WARP graph}
Fix a singleton payload set $P=\{\bullet\}$ and let $X_0=\Atom(\bullet)$ be the
unique atom.

\paragraph{First layer.} Let $S_1$ have vertices $\{u,v\}$ and a single edge
$e : u \to v$. Attach atoms everywhere:
\[
  \alpha_1(u)=X_0,\quad \alpha_1(v)=X_0,\quad \beta_1(e)=X_0,
\]
and set $X_1=(S_1,\alpha_1,\beta_1)$. By Definition~\ref{def:depth},
$\depthOf(X_1)=1$.

\paragraph{Second layer.} Let $S_2$ have vertices $\{w_1,w_2,w_3\}$ and edges
$e_1 : w_1 \to w_2$ and $e_2 : w_2 \to w_3$. Define
\[
  \alpha_2(w_1)=X_0,\quad \alpha_2(w_2)=X_1,\quad \alpha_2(w_3)=X_0,\qquad
  \beta_2(e_1)=\beta_2(e_2)=X_0,
\]
and put $X_2=(S_2,\alpha_2,\beta_2)$. Then
\[
  \depthOf(X_0)=0,\quad \depthOf(X_1)=1,\quad \depthOf(X_2)=2,
\]
since the deepest attachment is the inner $X_1$ at $w_2$.

In the unfolding tower, $\mathsf{unf}_1(X_2)$ collapses the inner $X_1$ at
$w_2$ to an atom, while $\mathsf{unf}_2(X_2)=X_2$, giving a concrete witness of
Definitions~\ref{def:warp-graph} and~\ref{def:depth}.

\clearpage
\addcontentsline{toc}{section}{References}
\bibliographystyle{alpha}
\bibliography{refs}

\end{document}
