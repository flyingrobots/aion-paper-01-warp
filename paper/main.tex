\documentclass[11pt]{article}

% Page layout: compact, readable margins
\usepackage[margin=1in]{geometry}

% Basic math and formatting packages
\usepackage{amsmath,amssymb,amsthm,mathtools}
\usepackage{enumitem}
\usepackage{graphicx}
\usepackage{xcolor}
\usepackage{tikz}
\usepackage{url}
\usepackage[colorlinks=true,linkcolor=blue,citecolor=blue,urlcolor=blue]{hyperref}

% Theorem environments
\newtheorem{definition}{Definition}
\newtheorem{proposition}{Proposition}
\newtheorem{example}{Example}

% Notation shortcuts (guarded to avoid clashes if already defined).
% Use serif glyphs; stay italic whether in text or math, even inside italic theorems.
\DeclareRobustCommand{\RMG}{\ifmmode\mathit{RMG}\else\textit{RMG}\fi}
\ifdefined\Atom\else\newcommand{\Atom}{\operatorname{Atom}}\fi
\ifdefined\depth\else\newcommand{\depth}{\operatorname{depth}}\fi
\ifdefined\OGraph\else\newcommand{\OGraph}{\mathbf{OGraph}}\fi
\ifdefined\OGraphT\else\newcommand{\OGraphT}{\mathbf{OGraph}_T}\fi
\ifdefined\Graph\else\newcommand{\Graph}{\mathbf{Graph}}\fi

% Local macros and wordmarks
% Macros for the WARPs paper
% Shared commands to keep notation consistent across the manuscript.
\usepackage{tikz} % Needed for \AIONLogo in this macros file
\usetikzlibrary{positioning,calc,shapes.geometric}
\newcommand{\AION}{\textrm{AI}\ensuremath{\Omega}\textrm{N}}
\newcommand{\AIONProjectURL}{\url{https://github.com/flyingrobots/aion}}


\title{Recursive Metagraphs}
\author{James Ross}
\date{December 2025}

\begin{document}

\maketitle

\begin{abstract}
Many computational systems are best described as graphs of graphs of graphs:
program call graphs whose nodes carry abstract syntax trees, distributed systems
whose edges carry communication traces, or world models whose entities themselves
contain rich internal structure.  Ordinary graphs and hypergraphs treat these
layers separately.  This paper isolates a single, uniform object---the
\emph{recursive metagraph} (RMG)---that captures this nested structure
categorically.

An RMG is a finite directed multigraph whose vertices and edges carry, as
attachments, further RMGs, forming a well-founded tree of graph-shaped
objects.  We give an inductive definition and an equivalent initial-algebra
presentation, define morphisms and the category \textbf{RMG}, and show how
ordinary typed graphs and hypergraphs embed fully and faithfully.  The
construction is deliberately minimal: it is just expressive enough to serve as
the state object for later work on DPO rewriting, deterministic concurrent
semantics, holographic provenance, and observer geometry in the AI$\Omega$N
Foundations series, while remaining simple enough to admit standard structural
recursion and induction principles.

This paper therefore does one job: to put recursive metagraphs on a clean,
categorical footing so that later semantics and systems papers can treat them
as a fixed base object rather than an informal data structure.
\end{abstract}

\section{Introduction}

Complex software does not live in a single flat graph.  A compiler juggles
syntax trees, control-flow graphs, and optimisation traces.  A database query
planner tracks schemas, query plans, and execution traces.  An AI agent may
carry a world model, internal goals, and a ledger of prior interactions.  In
practice, engineers wire these together by hand as graphs of graphs of graphs.

The usual mathematical tools---directed graphs, hypergraphs, typed open
graphs---are excellent for flat structure, but awkward for nested structure.
Every project invents its own ad hoc notion of ``graph with attached
subgraphs'', usually encoded as a pile of pointers and conventions.  That makes
it hard to transport results from one setting to another, and even harder to
state semantics that talk about the whole stack at once.

This paper extracts a minimal canonical object for this job: the
\emph{recursive metagraph} (RMG).  Informally, you take an ordinary finite
directed multigraph (the \emph{skeleton}) and allow each vertex and edge to
carry its own attached RMG, and those attachments may themselves have
attachments, and so on, but with the guarantee that this nesting is finite and
well-founded.

There are many formalisms in the same neighbourhood (nested graphs, bigraphs,
hierarchical state machines, etc.).  The design choice here is ruthlessly
pragmatic:

\begin{itemize}
  \item use as little structure as possible beyond standard typed graphs;
  \item make the construction compatible with adhesive categories and DPO
        graph rewriting~\cite{EL97,LS08}, so that rewriting extends cleanly;
  \item support both an inductive definition and an initial-algebra view, so
        that structural recursion and induction are available out of the box.
\end{itemize}

Everything else in the AI$\Omega$N Foundations series sits on top of this
object.  Later papers will use RMGs as the state space for deterministic
multiway rewriting, holographic provenance, rulial distance between observers,
and operating-system-level architecture.  Those topics are deliberately kept
offstage here; this paper focuses entirely on the definition and basic
categorical properties of RMGs.

\paragraph{Contributions.}
The specific contributions are:

\begin{enumerate}
  \item A precise inductive definition of recursive metagraphs as the least
        class generated by atomic payloads and decorated skeleton graphs.

  \item An equivalent initial-algebra presentation via a finitary polynomial
        endofunctor, yielding canonical recursion and induction schemes.

  \item A notion of morphism making RMGs into a category \textbf{RMG}, together
        with a faithful forgetful functor to the category of finite directed
        multigraphs.

  \item A relation to standard models: ordinary typed open graphs and
        hypergraphs embed fully and faithfully into suitable RMGs, so existing
        graph-based semantics extend without change.
\end{enumerate}

\paragraph{Structure of the paper.}
Section~\ref{sec:rmg-inductive} gives the inductive definition and some running
examples.  Section~\ref{sec:rmg-initial} recasts RMGs as an initial algebra.
Section~\ref{sec:rmg-unfold} introduces finite unfoldings and recursion schemes.
Section~\ref{sec:rmg-morphisms} defines morphisms and the category
\textbf{RMG}.  Section~\ref{sec:rmg-relation} relates RMGs to ordinary graphs
and hypergraphs and sketches how this sets the stage for DPO rewriting in later
work.  Section~\ref{sec:rmg-notation} summarises the main notation.

\section{Recursive Metagraphs}
\label{sec:rmg}

In this section we define recursive metagraphs and relate them to standard
graph models.  Throughout, a \emph{directed multigraph} $S$ is a tuple
$(V_S, E_S, s_S, t_S)$ with finite sets of vertices $V_S$ and edges $E_S$ and
source and target maps
\[
  s_S, t_S : E_S \to V_S.
\]
We write $V(S)$ and $E(S)$ when the underlying graph is clear from context.

Intuitively, an RMG has two layers:

\begin{itemize}
  \item a finite skeleton graph $S$ that describes the coarse shape of the
        state; and
  \item an \emph{attachment} at each vertex and edge, which is itself an RMG,
        capturing nested structure or payload.
\end{itemize}

Iterating attachments produces a finite, well-founded tree of graphs.  The rest
of the section spells this out.

\subsection{Inductive definition}
\label{sec:rmg-inductive}

Fix a set $P$ of \emph{atomic payloads}: literals, external identifiers, or
other opaque data items.  Think of $P$ as the stuff we are not going to model
internally (bytestrings, floats, external object IDs, \dots).

\begin{definition}[Recursive metagraph]
The class $\RMG$ of recursive metagraphs is the least set closed under the
following two constructors:
\begin{enumerate}
  \item For each $p \in P$ there is an \emph{atom} $\Atom(p) \in \RMG$.

  \item If $S = (V_S, E_S, s_S, t_S)$ is a finite directed multigraph and
        $\alpha : V_S \to \RMG$ and $\beta : E_S \to \RMG$ are attachment
        assignments, then the triple
        \[
          (S, \alpha, \beta) \in \RMG.
        \]
\end{enumerate}
We call $S$ the \emph{skeleton} and $\alpha$ and $\beta$ the \emph{vertex} and
\emph{edge} attachments.
\end{definition}

This is a standard well-founded inductive definition: every RMG is either a
bare atom or a finite skeleton whose vertices and edges carry smaller RMGs.
Because the construction is least, every RMG can be \emph{built up} from atoms
by finitely many applications of the second clause.

\begin{example}[Program call graph with attached syntax and provenance]
Consider a tiny program with functions $f$ and $g$ and a single call
$f \to g$.  The skeleton graph $S$ has two vertices $v_f, v_g$ and one edge
$e_{\mathrm{call}} : v_f \to v_g$.  Attach to $v_f$ and $v_g$ the abstract
syntax trees of the corresponding functions, and attach to $e_{\mathrm{call}}$
a small provenance graph recording optimisation choices or profiling data.

This yields an RMG $(S, \alpha, \beta)$ in which the coarse call graph,
per-function syntax, and per-call provenance all live in the same object.  The
attachments can themselves carry attachments, so syntax trees or provenance
graphs may be recursively structured without any change to the definition.
\end{example}

It is often convenient to write an RMG informally as
\[
  U = (G; \alpha, \beta),
\]
with $G$ emphasising the skeleton and $(\alpha, \beta)$ the attachments.

\subsection{Initial algebra viewpoint}
\label{sec:rmg-initial}

The inductive definition above already gives us a notion of recursion, but it is
useful to rephrase RMGs as an initial algebra for a polynomial functor.  This
makes structural recursion and induction theorems essentially automatic.

Let $\mathcal{G}$ be a small set of isomorphism-class representatives of finite
directed multigraphs (the allowed skeleton shapes).  Define a finitary
endofunctor
\[
  F : \mathbf{Set} \to \mathbf{Set}
\]
by
\[
  F(X) \;=\; P \;\;+\!\!\!\!\sum_{S \in \mathcal{G}}\!\! (V_S \to X) \times (E_S \to X).
\]
An element of $F(X)$ is either an atom in $P$ or a choice of skeleton shape
$S \in \mathcal{G}$ together with an $X$-label for each vertex and edge of $S$.

\begin{proposition}
Up to isomorphism, $\RMG$ is the carrier of the initial $F$-algebra.
\end{proposition}

Sketching the idea: the inductive clauses for $\RMG$ are exactly the clauses of
an $F$-algebra structure $\mathit{in} : F(\RMG) \to \RMG$.  Conversely, given any
$F$-algebra $(X, a)$, there is a unique \emph{fold}
$\mathit{fold}_a : \RMG \to X$ that interprets atoms and skeletons according to
$a$.  This is the usual initial-algebra story for polynomial functors.

Practically, this tells you that to define a function out of RMGs, it suffices
to say:

\begin{itemize}
  \item how it acts on atoms, and
  \item given a skeleton $S$ and recursively computed results for all
        attachments, how to combine those into a result for $(S, \alpha, \beta)$.
\end{itemize}

The result is then unique.  We will quietly rely on this pattern when defining
operations on RMGs later.

\subsection{Unfoldings and recursion schemes}
\label{sec:rmg-unfold}

Although RMGs are finite by definition, the attachment tree can be deep.  It is
often useful to reason about them via finite-depth approximations.

\begin{definition}[Depth]
The \emph{depth} of an RMG $X$ is the length of the longest attachment chain in
$X$.  Atoms have depth~0.  If $X = (S, \alpha, \beta)$, its depth is
\[
  \max\bigl\{ 1 + \depth(\alpha(v)), \; 1 + \depth(\beta(e))
    \mid v \in V_S, e \in E_S \bigr\}.
\]
\end{definition}

For $k \in \mathbb{N}$, define the \emph{$k$-unfolding} $\mathsf{unf}_k(X)$ to
be the RMG obtained by:

\begin{itemize}
  \item keeping all structure at attachment depths $0,\dots,k-1$ unchanged; and
  \item replacing every attachment at depth $\ge k$ by a fresh atom that
        remembers only that ``something'' was there.
\end{itemize}

This can be defined by structural recursion on $X$.  Each $\mathsf{unf}_k(X)$
is a finite RMG, and for $k \le \ell$ there is a natural embedding
$\mathsf{unf}_k(X) \to \mathsf{unf}_\ell(X)$.

The \emph{infinite unfolding} $\mathsf{unf}_\infty(X)$ is the directed colimit
of the diagram
\[
  \mathsf{unf}_0(X) \to \mathsf{unf}_1(X) \to \mathsf{unf}_2(X) \to \cdots
\]
in $\mathbf{Set}$ (or in $\mathbf{RMG}$ if you prefer).  Intuitively it records,
for each finite depth, what $X$ looks like up to that depth.  Every finite
pattern that occurs in $X$ appears in some finite unfolding, so reasoning about
$X$ can often be reduced to reasoning about all its approximants.

We will not need any stronger universal property of $\mathsf{unf}_\infty(X)$ in
this paper, but the picture is useful when thinking about observers and
multiway systems in later work.

\subsection{Morphisms and the category of recursive metagraphs}
\label{sec:rmg-morphisms}

To use RMGs as honest mathematical objects, we need a notion of morphism.

First regard the set $P$ of atomic payloads as a discrete category $\mathbf{P}$:
objects are elements of $P$, the only morphisms are identities.  We embed
atoms into our category faithfully: there are no non-trivial maps between
distinct atoms.

\begin{definition}[RMG morphism]
A \emph{morphism} $f : X \to Y$ of recursive metagraphs is defined by structural
recursion on depth.

\begin{itemize}
  \item On atoms: a morphism $\Atom(p) \to \Atom(p')$ exists iff $p = p'$, in
        which case it is the identity.

  \item On composite objects: for
        $X = (S, \alpha, \beta)$ and $Y = (S', \alpha', \beta')$, a morphism
        $f : X \to Y$ consists of
        \begin{enumerate}
          \item a graph homomorphism of skeletons
                \[
                  f_V : V_S \to V_{S'}, \qquad f_E : E_S \to E_{S'}
                \]
                preserving sources and targets; and

          \item for every vertex $v \in V_S$ a morphism of attachments
                $f_v : \alpha(v) \to \alpha'(f_V(v))$, and for every edge
                $e \in E_S$ a morphism
                $f_e : \beta(e) \to \beta'(f_E(e))$.
        \end{enumerate}
\end{itemize}

Composition and identities are defined componentwise.
\end{definition}

The recursion in the second clause is well-founded because attachments have
strictly smaller depth than the composite they sit in.

\begin{proposition}
Recursive metagraphs and their morphisms form a category, written
$\mathbf{RMG}$.
\end{proposition}

There is a forgetful functor
\[
  \pi : \mathbf{RMG} \to \mathbf{Graph}
\]
sending $(S, \alpha, \beta)$ to its skeleton $S$ and acting on morphisms by
forgetting the attachment components.  This functor will later be refined to a
fibration when we talk about rewriting on the ``skeleton plane'' and
``attachment planes''; for now it simply records that every RMG lives over an
ordinary finite directed multigraph.

\subsection{Relation to ordinary and hypergraphs}
\label{sec:rmg-relation}

RMGs are intended to \emph{extend} rather than replace standard graph models.
Two simple embeddings are particularly useful.

\paragraph{Ordinary graphs as shallow RMGs.}
Fix a set of atomic payloads $P$ with at least one distinguished element
$p_\bullet$.  Every finite directed multigraph $S$ can be viewed as a recursive
metagraph by attaching the constant atom $\Atom(p_\bullet)$ to every vertex and
edge:
\[
  \iota_{\mathrm{graph}}(S)
    = (S, \alpha, \beta)
\]
with $\alpha(v) = \Atom(p_\bullet)$ and
$\beta(e) = \Atom(p_\bullet)$ for all $v,e$.  Morphisms of graphs lift in the
obvious way.  This gives a fully faithful embedding of the category of finite
directed multigraphs into $\mathbf{RMG}$ as the subcategory of depth-$1$
objects with trivial attachments.

\paragraph{Hypergraphs via typed open graphs.}
Let $\OGraphT$ be a category of $T$-typed open graphs as in
Lack--Soboci\'nski~\cite{LS08}: objects are cospans $I \hookrightarrow G \hookleftarrow O$
of monomorphisms and morphisms are commuting maps of cospans.  This category is
adhesive, and typed hypergraphs embed fully and faithfully via the usual
incidence construction~\cite{EL97}.  In particular, DPO rewriting is
well-behaved in $\OGraph_T$.

We can regard an RMG whose skeleton lies in $\OGraphT$ as a ``recursive typed
open graph'': the base graph has open inputs and outputs, and attachments live
in the fibres over nodes and edges.  The forgetful functor
$\pi : \mathbf{RMG} \to \OGraphT$ simply drops attachments.  In the companion
paper on deterministic multiway semantics, we will exploit this to lift DPO
rewriting from skeletons to full RMG states.

For the present paper, the key point is that RMGs \emph{subsume} ordinary typed
open graphs and hypergraphs while adding recursively nested structure.  Any
model expressible in the usual DPO setting can be expressed as a shallow RMG;
models that genuinely need nested graphs get additional expressive power with
no change to the underlying graph-theoretic machinery.

\subsection{Notation summary}
\label{sec:rmg-notation}

For reference, we collect the main notation used in this paper.

\begin{center}
\begin{tabular}{ll}
  Symbol & Meaning \\ \hline
  $P$ & set of atomic payloads \\
  $\Atom(p)$ & atomic RMG corresponding to $p \in P$ \\
  $S = (V_S, E_S, s_S, t_S)$ & finite directed multigraph (skeleton) \\
  $(S, \alpha, \beta)$ & RMG with vertex and edge attachments \\
  $\RMG$ & class of all recursive metagraphs \\
  $\mathbf{RMG}$ & category of recursive metagraphs and morphisms \\
  $\depth(X)$ & depth of an RMG $X$ \\
  $\mathsf{unf}_k(X)$ & $k$-unfolding (finite-depth approximation) \\
  $\mathsf{unf}_\infty(X)$ & colimit of the unfolding tower of $X$ \\
  $\pi : \mathbf{RMG} \to \mathbf{Graph}$ & forgetful functor to skeleton graphs
\end{tabular}
\end{center}

Later papers in the AI$\Omega$N Foundations series will reuse this notation and
add symbols for rewrite rules, provenance payloads, wormholes, and observer
geometry.  In this first paper we deliberately stop at the level of static
structure.

\bibliographystyle{alpha}
\bibliography{refs}

\end{document}
